\documentclass[11pt]{scrartcl}
\usepackage[sexy, hints]{evan}
\usepackage{asymptote}

\begin{document}
\title{Chapter 7 Reference}
\date{\today}
\maketitle

\section{Basic Theorems}
\begin{theorem}
  [Barycentric Area Formula]
  Let $P_1$, $P_2$, $P_3$ be points with barycentric coordinates $P_i = (x_i, y_i, z_i)$ for $i = 1, 2, 3$. Then the signed area of $\triangle P_1P_2P_3$ is given by the determinant

  \[\frac{\left[P_1P_2P_3\right]}{\left[ABC\right]}=
    \left\lvert
    \begin{array}{ccc}
      x_1 & y_1 & z_1 \\
      x_2 & y_2 & z_2 \\
      x_3 & y_3 & z_3 \\
    \end{array}
    \right\rvert.
  \]

\end{theorem}

\begin{theorem}
  [Equation of a Line]
  The equation of a line takes the form $ux + vy + wz = 0$ where $u$, $v$, $w$ are real numbers. The $u$, $v$, and $w$ are unique up to scaling.
\end{theorem}

\begin{theorem}
  [Barycentric Cevian]
  Let $P = (x_1 : y_1 : z_1)$ be any point other than $A$. Then the points on line $AP$ (other than $A$) can be parametrized by
  \[(t:y_1:z_1)\]

  \noindent
  where $t\in \RR$ and $t+y_1+z_1 \neq 0$.

\end{theorem}

Coordinates of common points:
\begin{align*}
  G &= (1:1:1)\\
  I &= (a:b:c)\\
  I_A &= (-a:b:c)\\
  K &= (a^2:b^2:c^2)\\
  H &= (\tan A:\tan B:\tan C) = (S_BS_C:S_CS_A:S_AS_B)\\
  O &= (\sin 2A:\sin 2B:\sin 2C) = (a^2S_A:b^2S_B:c^2S_C)\\
  P_A &= (a^2 : 2S_A : 2S_A)\\
  Q_A &= (2S_A : b^2 : c^2)\\
  G_e &= ((s-b)(s-c) : (s-c)(s-a) : (s-a)(s-b))\\
  N &= ((s-a) : (s-b) : (s-c))
\end{align*}

\pagebreak
\section{Collinearity and Concurrence}
\begin{theorem}
  [Collinearity]
  Consider points $P_1$, $P_2$, $P_3$ with $P_i = (x_i : y_i : z_i)$ for $i = 1, 2, 3$. The three points are collinear if and only if
  \[0 =
    \left\lvert
    \begin{array}{ccc}
      x_1 & y_1 & z_1 \\
      x_2 & y_2 & z_2 \\
      x_3 & y_3 & z_3 \\
    \end{array}
    \right\rvert.
  \]
\end{theorem}

\begin{proposition}
  The line through two points $P = (x_1 : y_1 : z_1)$ and $Q = (x_2 : y_2 : z_2)$ is given precisely by the formula
    \[0 =
    \left\lvert
    \begin{array}{ccc}
      x & y & z \\
      x_1 & y_1 & z_1 \\
      x_2 & y_2 & z_2 \\
    \end{array}
    \right\rvert.
  \]
\end{proposition}

\begin{theorem}
  [Concurrence]
  Consider three lines
  \[\ell_i : u_ix + v_iy + w_iz = 0\]
  for i = 1, 2, 3. They are concurrent or all parallel if and only if
    \[0=
    \left\lvert
    \begin{array}{ccc}
      u_1 & v_1 & w_1 \\
      u_2 & v_2 & w_2 \\
      u_3 & v_3 & w_3 \\
    \end{array}
    \right\rvert.
  \]
\end{theorem}

\section{Displacement Vectors}
\begin{theorem}
  [Distance Formula]
  Let $P$ and $Q$ be two arbitrary points and consider a displacement vector $\ray{PQ} = (x, y, z)$. Then the distance from $P$ to $Q$ is given by
  \[\abs{PQ}^2=-a^2yz-b^2zx-c^2xy.\]
\end{theorem}

\begin{theorem}
  [Barycentric Circle]
  The general equation of a circle is
  \[-a^2yz-b^2zx-c^2xy+(ux+vy+wz)(x+y+z)=0\]
  for reals $u$, $v$, $w$.
\end{theorem}

\begin{theorem}
  [Barycentric Perpendiculars]
  Let $\ray{MN} = (x_1, y_1, z_1)$ and $\ray{PQ} = 
(x_2, y_2, z_2)$ be displacement vectors. Then $MN \perp PQ$ if and only if
\[0 = a^2(z_1y_2 + y_1z_2) + b^2(x_1z_2 + z_1x_2) + c^2(y_1x_2 + x_1y_2).\]
\end{theorem}

\section{Conway's Notations}
\begin{proposition}
  [Conway Identities]
  Let $S$ denote twice the area of triangle $ABC$, then
  \begin{align*}
    S^2 &= S_{AB} + S_{BC} + S_{CA}\\
        &= S_{BC} + a^2S_A\\
        &= \frac{1}{2}(a^2S_A + b^2S_B + c^2S_C)\\
        &= (bc)^2 - S_A^2.
  \end{align*}
  In particular,
  \[a^2S_A + b^2S_B - c^2S_C = 2S_{AB}.\]
\end{proposition}
\section{Power of a Point}
\begin{lemma}
  [Barycentric Power of a Point]
  \[\Pow_\omega(P) = - a^2yz - b^2zx - c^2xy + (x + y + z)(ux + vy + wz).\]
\end{lemma}

\begin{lemma}
  [Barycentric Radical Axis]
  Suppose two non-concentric circles are given by the equations
  \begin{align*}
    - a^2yz - b^2zx - c^2xy + (x + y + z)(u_1x + v_1y + w_1z) &= 0\\
    - a^2yz - b^2zx - c^2xy + (x + y + z)(u_2x + v_2y + w_2z) &= 0.
  \end{align*}
  Then their radical axis is given by
  \[(u_1 - u_2)x + (v_1 - v_2)y + (w_1 - w_2)z = 0.\]
\end{lemma}
\begin{lemma}
  The tangent to ($ABC$) at $A$ is given by
  \[b^2z + c^2y = 0.\]
  center
\end{lemma}

\begin{figure}[ht]
\centering

  \begin{asy}
/*
Converted from GeoGebra by User:Azjps using Evan's magic cleaner
https://github.com/vEnhance/dotfiles/blob/main/py-scripts/export-ggb-clean-asy.py
*/
pair A = (-0.47070,0.88228);
pair B = (-0.99083,-0.13508);
pair C = (0.98897,-0.14810);
pair Q = (-0.62342,-0.13750);
pair P = (-0.33141,-0.13942);
pair N = (-0.77614,-1.15729);
pair M = (-0.19211,-1.16113);
pair X = (-0.37405,-0.92740);

import graph;
size(8cm);
pen qqwuqq = rgb(0.,0.39215,0.);
draw(arc(C,0.08681,144.78165,179.62332)--C--cycle, linewidth(0.6) + qqwuqq);
draw(arc(A,0.08681 * 1.5,-117.07793,-82.23626)--A--cycle, linewidth(0.6) + qqwuqq);
draw(arc(B,0.08681,-0.37667,62.92206)--B--cycle, linewidth(0.6) + red);
draw(arc(A,0.08681,-98.51708,-35.21834)--A--cycle, linewidth(0.6) + red);
draw(circle((0.,0.), 1.), linewidth(0.6));
draw(A--B, linewidth(0.6));
draw(B--C, linewidth(0.6));
draw(C--A, linewidth(0.6));
draw(B--M, linewidth(0.6));
draw(C--N, linewidth(0.6));
draw(A--N, linewidth(0.6));
draw(A--M, linewidth(0.6));

dot("$A$", A, dir(64));
dot("$B$", B, dir(160));
dot("$C$", C, dir(20));
dot("$Q$", Q, dir(46));
dot("$P$", P, dir(65));
dot("$N$", N, dir(153));
dot("$M$", M, dir(46));
dot("$X$", X, dir(89));
\end{asy}
\caption{IMO 2014/4}
\end{figure}

\begin{figure}[ht]
\centering
\begin{asy}
/*
Converted from GeoGebra by User:Azjps using Evan's magic cleaner
https://github.com/vEnhance/dotfiles/blob/main/py-scripts/export-ggb-clean-asy.py
*/
pair A = (-0.38517,0.92284);
pair B = (-0.84127,-0.54061);
pair C = (0.84462,-0.53536);
pair D = (-0.42158,-0.53930);
pair F = (0.12314,0.32010);
pair E = (-0.70393,-0.09994);
pair M = (0.00167,-0.53798);
pair P_A = (-0.11869,-0.08345);

import graph;
size(10cm);
draw(A--B, linewidth(0.6));
draw(B--C, linewidth(0.6));
draw(C--A, linewidth(0.6));
draw(circle((-0.63372,0.19750), 0.76674), linewidth(0.6) + dotted);
draw(circle((0.20929,0.17651), 0.95415), linewidth(0.6) + dotted);
draw(circle((-0.42443,0.37401), 0.55023), linewidth(0.6));
draw(A--M, linewidth(0.6) + red);

dot("$A$", A, dir(62));
dot("$B$", B, dir(64));
dot("$C$", C, dir(62));
dot("$D$", D, dir(65));
dot("$F$", F, dir(28));
dot("$E$", E, dir(27));
dot("$M$", M, dir(64));
dot("$P_A$", P_A, dir(65));
\end{asy}
\caption{ELMO Shortlist 2013}
\end{figure}

\begin{center}
\begin{asy}
/*
Converted from GeoGebra by User:Azjps using Evan's magic cleaner
https://github.com/vEnhance/dotfiles/blob/main/py-scripts/export-ggb-clean-asy.py
*/
pair A = (-0.65711,0.75379);
pair B = (-0.85811,-0.51346);
pair C = (0.84462,-0.53536);
pair D = (-0.43423,-0.51891);
pair F = (0.00966,0.18140);
pair E = (-0.76998,0.04218);
pair M = (-0.00674,-0.52441);
pair H = (-0.67060,-0.29503);
pair Op = (-0.01348,-1.04882);
pair Hp = (0.64362,-1.80261);
pair P_A = (-0.23597,-0.07389);
pair O = (0.,0.);

import graph;
size(12cm);
draw(circle(O, 1.), linewidth(0.6));
draw(A--B, linewidth(0.6));
draw(B--C, linewidth(0.6));
draw(C--A, linewidth(0.6));
draw(circle((-0.63823,0.10122), 0.65283), linewidth(0.6) + dotted);
draw(circle((0.21519,0.25068), 1.00699), linewidth(0.6) + dotted);
draw(circle((-0.42303,0.35191), 0.46508), linewidth(0.6));
draw(circle(Op, 1.), linewidth(0.6));
draw(A--Hp, linewidth(0.6) + red);
draw(H--P_A, linewidth(0.6));
draw(Hp--H, linewidth(0.6));
draw(A--H, linewidth(0.6));
draw(O--Op, linewidth(0.6));

dot("$A$", A, dir(61));
dot("$B$", B, dir(63));
dot("$C$", C, dir(62));
dot("$D$", D, dir(62));
dot("$F$", F, dir(27));
dot("$E$", E, dir(26));
dot("$M$", M, dir(67));
dot("$H$", H, dir(64));
dot("$O'$", Op, dir(63));
dot("$H'$", Hp, dir(65));
dot("$P_A$", P_A, dir(62));
dot("$O$", O, dir(65));
\end{asy}
\end{center}
\end{document}
